\chapter{代謝制御解析}
代謝経路の各酵素が流束を「律速」する度合いを評価するための理論。エジン
バラ大学のKacserとBurns、ベルリン・フンボルト大学のHeinrichとRapoportに
よって1973年から74年にかけて独立に発表された。Metabolic Control
Analysisの頭文字を取ってMCAと略記される。

\section{流束制御係数(Flux Control Coefficient)}
流束制御係数とは、反応速度\(v_1\)が1\%増えたとき、流束Jは何\%増えるかを示す指標である。

\[C^J_{v_1} = \frac{v_1}{J}\frac{\partial J}{\partial v_!}\]

MCAの理論体系では、\(C^J_{v_i}=1\)である酵素を律速酵素(rate-limiting
enzyme)と定義する。同理論体系には、代謝経路内の基質や反応に関する指標が
複数種類あるが(後述)、流束制御係数は理論創案当初から最も重視されてい
る。

\section{流束制御係数の加法定理(summation theorem)}
\paragraph{定理}

\[\sum_{i=1}^n C^J_{v_i}=1\]

流束Jに影響するすべての\(C^J_{v_i}\)の総和は1(=100\%)。 
限られた\(C^J_{v_i}\)を複数の酵素で分け合っている。

\paragraph{証明}
流束の全微分を考える。
\[dJ = \frac{\partial J}{\partial v_1}dv_1 + \frac{\partial J}{\partial v_2}dv_2 + \cdots + \frac{\partial J}{\partial v_n}dv_n  \]

両辺Jで割り、右辺の各項に \( v_i / v_i (i = 1, \cdots , n)\)をかける。

\begin{eqnarray*}
\frac{dJ} {J} & = & \frac{v_1}{J}\frac{\partial J}{\partial v_1}\frac{dv_1}{v_1} + \frac{v_2}{J}\frac{\partial J}{\partial v_2}\frac{dv_2}{v_2}  + \cdots + \frac{v_n}{J}\frac{\partial J}{\partial v_n}\frac{dv_n}{v_n} \\
 & = & C^J_{v_1}\frac{dv_1}{v_1} + C^J_{v_2}\frac{dv_2}{v_2} + \cdots + C^J_{v_n}\frac{dv_n}{v_n}
\end{eqnarray*}

\(v_1\)~\(v_n\) が同じ割合\(\alpha\)だけ増えると、 Jも\(\alpha\)だけ増加する。
\[
\alpha = \frac{dJ} {J} = \frac{dv_1}{v_1} = \frac{dv_2}{v_2} = \cdots = \frac{dv_n}{v_n}
\]

よって、

\begin{eqnarray*}
\frac{dJ} {J}  & = & C^J_{v_1}\frac{dv_1}{v_1} + C^J_{v_2}\frac{dv_2}{v_2} + \cdots + C^J_{v_n}\frac{dv_n}{v_n}\\
\alpha & = & \alpha  C^J_{v_1} + \alpha C^J_{v_2} + \cdots + \alpha C^J_{v_n}
\end{eqnarray*}

両辺\(\alpha\)で割ると

\[1  =  C^J_{v_1} + C^J_{v_2} + \cdots + C^J_{v_n}\]

となり、加法定理が証明できた。

\section{3つの主要係数}
流束制御係数の他に、以下の2つの係数が重要である。

\paragraph{濃度制御係数(Concentration Control Coefficient)}
\[C^S_{v_1} = \frac{v_1}{[S]}\frac{\partial [S]}{\partial v_1}\]

反応速度\(v_1\)が1\%増えたとき基質\(S\)の定常状態濃度が何\%増えるかを示す指標。

\paragraph{弾力性係数(Elasticity)}

\[\varepsilon^{v_1}_{S} = \frac{[S]}{v_1}\frac{\partial v_1}{\partial [S]}\]
基質\(S\)が1\%増えたとき、反応速度\(v_1\)が何\%増えるかを示す指標。反応の次数を示す。


\section{結合定理(connectivity theorem)}
\paragraph{定義}
\[\sum_{i=1}^n C^J_{v_i}\varepsilon^{v_i}_{S}=0\]

酵素固有の性質で決まる弾力性係数は、代謝経路全体の中では局所的な要素と
言える。一方、流束は複数の酵素にまたがるので、代謝経路の大域的な性質で
ある。結合定理は、局所的指標である弾力性係数と、大域的指標である流束制
御係数を1本につなぐ指揮として重要である。

\paragraph{証明}
反応速度の全微分を考える。

\[dv_i = \frac{\partial v_i}{\partial E} dE + \frac{\partial v_i}{\partial S_j} dS_j\]

\(v_i \propto E\) より \(v_i = k E\) とおいて、上式を変形する。

\begin{eqnarray*}
\frac{dv_i}{v_i} & = &\frac{E}{v_i}\frac{\partial v_i}{\partial E}\frac{dE}{E} + \frac{S_j}{v_i}\frac{\partial v_i}{\partial S_j} \frac{dS_j}{S_j}\\
& = & \frac{1}{k} k \frac{dE}{E} + \varepsilon^{v_i}_{S_j} \frac{dS_j}{S_j}\\
& = & \frac{dE}{E} + \varepsilon^{v_i}_{S_j}\frac{dS_j}{S_j}
\end{eqnarray*}

酵素濃度増、基質濃度減による効果が打ち消しあって\(dv_i=0\)になる組み合
わせが存在する。

\begin{figure}[h]
\begin{center}
\includegraphics[scale=0.5]{../MCA/img/mca4.epsi}
\end{center}
\end{figure}

すなわち、先ほど導いた

\[\frac{dv_i}{v_i} = \frac{dE}{E} + \varepsilon^{v_i}_{S_j}\frac{dS_j}{S_j}\]

に\(dv_i=0\)を代入して

\[0  = \frac{dE}{E} + \varepsilon^{v_i}_{S_j}\frac{dS_j}{S_j}\]

次いで、

\[\frac{dJ} {J}   =  C^J_{v_1}\frac{dv_1}{v_1} + C^J_{v_2}\frac{dv_2}{v_2} + \cdots + C^J_{v_n}\frac{dv_n}{v_n}\]

に\(0  = \frac{dE}{E} + \varepsilon^{v_i}_{S_j}\frac{dS_j}{S_j}\)および\(dJ = dv_1 = ¥cdots = dv_n = 0\)を代入して

\[0 = C^J_{v_1} \left( -\varepsilon^{v_1}_{S_j} \frac{dS_j}{S_j}\right)+\cdots+C^J_{v_n} \left( -\varepsilon^{v_n}_{S_j} \frac{dS_j}{S_j}\right) \]

\[0 = C^J_{v_1} \varepsilon^{v_1}_{S_j} +\cdots+C^J_{v_n} \varepsilon^{v_n}_{S_j}\]

\section{基質濃度制御係数に関する定理}
証明は省略する。
\paragraph{加法定理}
\[\sum^{n}_{i=1} C^{S}_{v_i} = 0\]
\paragraph{結合定理}
\[\sum^{n}_{i=1} C^{S_j}_{v_i} \varepsilon^{v_i}_{S_k} = - \delta_{jk}\]

\subsection{係数の測定法}
ダブルモジュレーション法

\[dJ = \frac{\partial v_{\rm GPI}}{\partial \rm [G6P]} d{\rm [G6P]} + \frac{\partial v_{\rm GPI}}{\partial \rm [F6P]} d{\rm [F6P]}\]

\paragraph{演習}
ある細胞の解糖系についてダブルモジュレーション実験を行い、下記のようなデータを得た。 \(\varepsilon\)を求めなさい。

\begin{itemize}
\item コントロール
\begin{itemize}
\item [G6P]=80\(\mu\)M、[F6P]=12\(\mu\)M、\(v_{GPI}\)=2400\(\mu\)M/min 
\end{itemize}

\item \(\Delta\)[Glucose]=1.0mMのとき
\begin{itemize}
\item [G6P]=88\(\mu\)M、[F6P]=15\(\mu\)M、\(v_{GPI}\)=2440\(\mu\)M/min
\end{itemize}

\item \(\Delta\)[Glucose]=2.0mMのとき
\begin{itemize}
\item [G6P]=90\(\mu\)M、[F6P]=14\(\mu\)M、\(v_{GPI}\)=2520\(\mu\)M/min
\end{itemize}
\end{itemize}

\section{行列形式への拡張と基本方程式}
今までの議論は一直線の代謝経路のみを前提としていた。実際の代謝系にはいたるところに分岐点がある。分岐点のある経路でもMCAを使いたい。

\subsection{MCAの基本方程式}
\[
\left(
\begin{array}{c}
{\bf {}^{non}C^J}\\
{\bf {}^{non}C^S}
\end{array}
\right)
\left(
\begin{array}{cc}
{\bf K} & {\bf {}^{non}\varepsilon L}\\
\end{array}
\right)
=
\left(
\begin{array}{rr}
{\bf K} & {\bf 0}\\
{\bf 0} & {\bf -L}\\
\end{array}
\right)
\]

\({\bf C^J}\)、 \({\bf C^S}\) 、\({\varepsilon}\)はそれぞれの係数からなる行列。``non''は正規化されていないことを表す。
{\bf K}は零空間。{\bf L}はこれから紹介。

\paragraph{\({\bf C^J}\)、 \({\bf C^S}\) 、\({\varepsilon}\)の中身}
\[
{\bf C^J}
=
\left(
\begin{array}{cccc}
C^{J_1}_{v_1} & C^{J_1}_{v_2} & \cdots & C^{J_1}_{v_n} \\
C^{J_2}_{v_1} & C^{J_2}_{v_2} &        & C^{J_2}_{v_n} \\
\vdots       &              & \ddots & \vdots \\
C^{J_m}_{v_1} & C^{J_m}_{v_2} & \cdots & C^{J_m}_{v_n} \\
\end{array}
\right)
\begin{array}{c}
 \\
 \\
 \\
,\\
\end{array}
{\bf C^S}
=
\left(
\begin{array}{cccc}
C^{S_1}_{v_1} & C^{S_1}_{v_2} & \cdots & C^{S_1}_{v_n} \\
C^{S_2}_{v_1} & C^{S_2}_{v_2} &        & C^{S_2}_{v_n} \\
\vdots       &              & \ddots & \vdots \\
C^{S_k}_{v_1} & C^{S_k}_{v_2} & \cdots & C^{S_k}_{v_n} \\
\end{array}
\right)
\]
\[
{\boldsymbol \varepsilon}
=
\left(
\begin{array}{cccc}
\varepsilon^{v_1}_{S_1} & \varepsilon^{v_1}_{S_2} & \cdots & \varepsilon^{v_1}_{S_k} \\
\varepsilon^{v_2}_{S_1} & \varepsilon^{v_2}_{S_2} &        & \varepsilon^{v_2}_{S_k} \\
\vdots       &              & \ddots & \vdots \\
\varepsilon^{v_n}_{S_1} & \varepsilon^{v_n}_{S_2} & \cdots & \varepsilon^{v_n}_{S_k} \\
\end{array}
\right)
\]

\subsection{分岐のある経路の流束}
以下のように考える。
\begin{figure}[h]
\begin{center}
\includegraphics[scale=0.5]{../MCA/img/mca6.epsi}
\end{center}
\end{figure}


\subsection{Link matrix}
化学量論行列{\bf N}の線形独立な行を上側に寄せる

\[
{\bf N}
=
\left(
\begin{array}{l}
{\bf N^0} \\
{\bf N^{\prime}}
\end{array}
\right)
=
{\bf LN^0}
=
\left(
\begin{array}{l}
{\bf I_{rank({\bf N})}} \\
{\bf L^{\prime}}
\end{array}
\right)
{\bf N^0}
\]
上式の{\bf L}をLink matrixと呼ぶ。{\bf L}の行数は{\bf N}の行数と等しい。列数は rank({\bf N})である。
\paragraph{Link matrixの例}
\[
{\bf N}
=
\left(
\begin{array}{rrrr}
1 & -1 & 0 & 0\\
0 & 1  & -1 & 0\\
0 & -1 & 0 & 1\\
0 & 1 & 0 & -1 \\
\end{array}
\right)
\longrightarrow
{\bf N^0}
=
\left(
\begin{array}{rrrr}
1 & -1 & 0 & 0\\
0 & 1  & -1 & 0\\
0 & -1 & 0 & 1\\
\end{array}
\right)
\]

\({\bf N = LN^0}\)より

\[
{\bf L}
=
\left(
\begin{array}{rrrr}
1 & 0 & 0\\
0 & 1 & 0\\
0 & 0 & 1\\
0 & 0 & -1\\
\end{array}
\right)
\]

\subsection{基本方程式における加法・結合定理}
\[
\begin{array}{ccc}
\left(
\begin{array}{c}
{\bf {}^{non}C^J}\\
{\bf {}^{non}C^S}
\end{array}
\right)
\left(
\begin{array}{cc}
{\bf K} & {\bf {}^{non}\varepsilon L}\\
\end{array}
\right)
&
=
&
\left(
\begin{array}{rr}
{\bf K} & {\bf 0}\\
{\bf 0} & {\bf -L}\\
\end{array}
\right)\\
 & \downarrow & \\
\begin{array}{l}
\mbox{\bf 加法定理}\\
{\bf {}^{non}C^J K = K} \\
{\bf {}^{non}C^S K = 0} \\
\end{array}
&

&
\begin{array}{l}
\mbox{\bf 結合定理}\\
{\bf {}^{non}C^J}{\bf {}^{non}\varepsilon L = 0}\\
{\bf {}^{non}C^S}{\bf {}^{non}\varepsilon L = - L}\\
\end{array}
\end{array}
\]

\section{基本方程式の証明}
\subsection{準備1: 2変数の合成関数の微分公式}
\(z=f(x,y)\) において \(x=g(u,v)\)、\(y=h(u,v)\)で表されるとき、

\begin{eqnarray*}
\frac{\partial z}{\partial u} & = &  \frac{\partial z}{\partial x}\frac{\partial x}{\partial u} + \frac{\partial z}{\partial y}\frac{\partial y}{\partial u}   \\
\frac{\partial z}{\partial v} & = &  \frac{\partial z}{\partial x}\frac{\partial x}{\partial v} + \frac{\partial z}{\partial y}\frac{\partial y}{\partial v}   
\end{eqnarray*}

\subsection{準備2: \({\bf C^S}\)の行列表記}
Nv(S(E),E)=0をEで偏微分。「準備1」の微分公式より

\[
{\bf N}\left(\frac{\partial \bf v}{\partial \bf S}\frac{\partial \bf S}{\partial \bf E}+\frac{\partial \bf v}{\partial \bf E}\frac{\partial \bf E}{\partial \bf E}\right)=0
\]

移項して

\[
{\bf N}\frac{\partial \bf v}{\partial \bf S}\frac{\partial \bf S}{\partial \bf E}=-{\bf N}\frac{\partial \bf v}{\partial \bf E}
\]

\[
\frac{\partial \bf S}{\partial \bf E}=-\left({\bf N}\frac{\partial \bf v}{\partial \bf S}\right)^{-1}{\bf N}\frac{\partial \bf v}{\partial \bf E}
\]

両辺に\(\frac{\partial \bf E}{\partial \bf v}\)をかけて、

\[
{\bf {}^{non}C^S}=\frac{\partial \bf S}{\partial \bf v}=-({\bf N {}^{non}{\boldsymbol \varepsilon}})^{-1}{\bf N}
\]

\subsection{準備3: \({\bf C^J}\)の行列表記}
J=v(S(E),E)をEで偏微分

\[\frac{\partial \bf J}{\partial \bf E}=\frac{\partial \bf v}{\partial \bf S}\frac{\partial \bf S}{\partial \bf E}+\frac{\partial \bf v}{\partial \bf E}\frac{\partial \bf E}{\partial \bf E}\]

両辺に \(\frac{\partial \bf E}{\partial \bf v}\) をかけて

%\begin{eqnarray*}
\[
\frac{\partial \bf J}{\partial \bf v} = {\bf {}^{non}{\boldsymbol \varepsilon}}\frac{\partial \bf S}{\partial \bf v}+{\bf I} \]
\[{\bf {}^{non}C^J =  {}^{non}{\boldsymbol \varepsilon}{}^{non}C^S+I}
\]%\end{eqnarray*}

\subsection{結合定理の証明}
\({\bf {}^{non}C^S}=-({\bf N {}^{non}{\boldsymbol \varepsilon}})^{-1}{\bf N}\)の両辺に\({}^{\bf non}{\boldsymbol \varepsilon}{\bf L}\)をかけて

\begin{eqnarray*}
{\bf {}^{non}C^S{}^{non}{\boldsymbol \varepsilon} L}&=&-({\bf N {}^{non}{\boldsymbol \varepsilon}})^{-1}{\bf N{}^{non}{\boldsymbol \varepsilon} L}\\
&=& {\bf -L}
\end{eqnarray*}

\({\bf {}^{non}C^J={}^{non}{\boldsymbol \varepsilon}{}^{non}C^S+I}\)の両辺に\({}^{\bf non}{\boldsymbol \varepsilon}{\bf L}\)をかけて

\begin{eqnarray*}
{\bf {}^{non}C^J{}^{non}{\boldsymbol \varepsilon} L } & = & {\bf {}^{non}{\boldsymbol \varepsilon}{}^{non}C^S{}^{non}{\boldsymbol \varepsilon} L+{}^{non}{\boldsymbol\varepsilon} L}\\
& = & {\bf {}^{non}{\boldsymbol \varepsilon}(-L)+{}^{non}{\boldsymbol \varepsilon} L}\\
& = & 0
\end{eqnarray*}

これで結合定理の証明が完了した。残るは加法定理。

\subsection{加法定理の証明}
\({\bf {}^{non}C^S}=-({\bf N {}^{non}{\boldsymbol \varepsilon}})^{-1}{\bf N}\)の両辺に\({\bf K}\)をかけて

\begin{eqnarray*}
{\bf {}^{non}C^SK} & = &-({\bf N {}^{non}{\boldsymbol \varepsilon}})^{-1}{\bf NK}\\
& = & {\bf 0}
\end{eqnarray*}

\({\bf {}^{non}C^J={}^{non}{\boldsymbol \varepsilon}{}^{non}C^S+I}\)の両辺に\({\bf K}\)をかけて

\begin{eqnarray*}
{\bf {}^{non}C^JK} & = & {\bf {}^{non}{\boldsymbol \varepsilon}{}^{non}C^S K+K}\\
& = & {\bf K}
\end{eqnarray*}

以上で加法定理を証明できた。

\subsection{基本方程式の使い方}
\[
\left(
\begin{array}{c}
{\bf {}^{non}C^J}\\
{\bf {}^{non}C^S}
\end{array}
\right)
\left(
\begin{array}{cc}
{\bf K} & {\bf {}^{non}\varepsilon L}\\
\end{array}
\right)
=
\left(
\begin{array}{rr}
{\bf K} & {\bf 0}\\
{\bf 0} & {\bf -L}\\
\end{array}
\right)
\]

K、LはNから求められる。\({\bf {}^{non}\varepsilon}\)、\({\bf {}^{non}C^J}\)、\({\bf {}^{non}C^S}\) はどれか1つがわかっていればあとは基本方程式から求まる。最後に\({\bf {}^{non}\varepsilon}\)、\({\bf {}^{non}C^J}\)、\({\bf {}^{non}C^S}\) を正規化する。これで、分岐のある代謝経路でもMCAが可能になる。

\subsection{正規化の方法}
対角行列を左右からかける。

\begin{eqnarray*}
{\bf C^J} & = & ({\rm diag}\ {\bf J})^{-1}({\bf {}^{non}C^J})({\rm diag}\ \bf J)\\
{\bf C^S} & = & ({\rm diag}\ {\bf S})^{-1}({\bf {}^{non}C^S})({\rm diag}\ \bf J)\\
{\boldsymbol \varepsilon} & = & ({\rm diag}\ {\bf J})^{-1}({\bf {}^{non}\boldsymbol \varepsilon})({\rm diag}\ \bf S)
\end{eqnarray*}

以下に対角行列の例を示す。

\[
{\bf J}=\left(
\begin{array}{r}
J_1\\
J_2\\
J_3
\end{array}
\right)
\longrightarrow
{\rm diag}\ {\bf J}=\left(
\begin{array}{ccc}
J_1 & 0 & 0 \\
0   & J_2 & 0\\
0   & 0   & J_3
\end{array}
\right)
,
({\rm diag}\ {\bf J})^{-1}=\left(
\begin{array}{ccc}
\frac{1}{J_1} & 0 & 0 \\
0   & \frac{1}{J_2} & 0\\
0   & 0   & \frac{1}{J_3}
\end{array}
\right)
\]
\subsection{演習}
下のような代謝経路について、下記の問いに答えなさい
\begin{figure}[h]
\begin{center}
\includegraphics[scale=0.5]{../MCA/img/mca5.epsi}
\end{center}
\end{figure}

\begin{enumerate}
\item Link matrix が1(スカラー)であることを示しなさい。
\item {\bf K}行列を求めなさい。
\item 以下の条件のもとで\({\bf C^J}\)行列を求めなさい。正規化すること。\\
\[
{}^{\bf non}{\boldsymbol \varepsilon}=\left(
\begin{array}{r}
-0.4\\
0.4\\
0.2\\
\end{array}
\right)
{\rm min}^{-1}
\begin{array}{c}
 \\
 \\
,\\
\end{array}
\left(
\begin{array}{c}
v_1\\
v_2\\
v_3
\end{array}
\right)
=
\left(
\begin{array}{r}
1.0\\
0.6\\
0.4\\
\end{array}
\right)
{\rm mM/min}
\]
\item 加法定理が成り立っていることを確認しなさい。
\end{enumerate}


\section{Further Reading}
清水先生の本

ステファノポーラス、アリスティド、ニールセン 
「代謝工学 原理と方法論」 東京電機大学出版局

Reinhart Heinrich and Stefan Schuster 
“The regulation of cellular systems”, Chapman and Hall

Edda Klipp et al. 
“Systems biology in practice”, Wiley VCH

David Fell
“Understanding the control of metabolism”, Portland press

