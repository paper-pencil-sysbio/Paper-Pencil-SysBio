\chapter{代謝流束均衡解析}
\paragraph{この手法で何がわかるか}
細胞内の代謝経路のうち、どのルートが活発に動いているかがわかります。

\paragraph{どのような数学を使うか}
行列

\begin{center}
{\bf 概要}
\end{center}

\begin{enumerate}
\item 代謝経路の構造は行列で書くことができます。これを{\bf 化学量論係数行列}と呼びます。
\item {\bf 定常状態}とは、着目した物質の生成速度と消費速度が等しい状態のことです
\item 定常状態における反応速度のことを{\bf 流束}と呼びます
\item 化学量論係数行列から、流束を求めることができます
\end{enumerate}

\section{代謝経路のどのルートを使うか}
\paragraph{代謝は化学反応の集まり} 
代謝 (metabolism) とは生命活動に必要な分子を合成し、用済みになった分子を
分解する化学反応[たち]のことです。原則的に、これらの化学反応は酵素タンパク
質によって触媒されることで起きています。みなさんも、高校の生物の授業で
解糖系やクエン酸回路といった代謝経路を構成する代謝物や酵素の名前を習っ
たかと思います。

\paragraph{高校生物で習うのは広大な代謝経路のごく一部} 
高校で習う解糖系やクエン酸回路は、細胞内で活動している代謝経路全体の一
部にすぎません(FIXME KEGGのglobal map)。細胞の3大エネルギー源である、
糖、アミノ酸、脂肪のうち、みなさんが高校生物で習ったのは糖からATPをつく
る「糖代謝」です。高校で習う糖代謝以外にも、アミノ酸、脂肪からエネルギー
を取り出す経路があります。また、細胞分裂する際には細胞の構成部品をすべ
て2倍に増やしておかなければなりません。この要求を満たすため
に、DNAやRNAの材料となる核酸を合成する反応経路や、細胞膜の材料であるリ
ン脂質を合成する経路があります。

\paragraph{代謝にかかわる酵素タンパク質は約FIXME個の遺伝子にコードされている}
これらの反応経路を構成している化学反応の合計は、大腸菌
で FIXME 個、ヒトで FIXME 個と見積もられています。これら1つ1つの化学反
応に対応して、その反応を進行させる酵素タンパク質の遺伝子が、DNAにコード
されています。たとえばヒトの解糖系を構成する 10 個の化学反応は、FIXME 個の酵素の遺伝子にコードされており、図 FIXME のように 14 個の染色体上にあります。

% glycolysis_enzyme.txt も参照のこと
%
% HK1 10, HK2 2, HK3 5, (HKDC1 10) , GCK (HK4) 7, ADP-GK 15
% GPI 19
% PFKL 21, PFKM 12, PFKP 10
% ALDOA 16, ALDOB 9, ALDOC 17
% TPI1 12
% GAPDH 12, GAPDHS 19
% PGK1 X, PGK2 6
% PGAM4 X, PGAM1 10, PGAM2 7
% ENO1 1, ENO2 12, ENO3 17
% PKLR 1, PKM 15


\paragraph{広大な代謝経路には、活動しているルートとそうでないルートがある}
これらの代謝経路はすべて同じように活動しているわけではありません。その
時々の環境によって、細胞に不足している物質を合成する経路の酵素タンパク
質はより多く合成され、化学反応の起こる回数がより多くなります。細胞に十
分足りている物質を作ってもエネルギーの無駄で、長い進化の過程でそういう
無駄はなるべくしないようになっています。そういう化学反応を触媒する酵素
タンパク質はあまり合成されず、その化学反応が起こる回数は少なくなります。
(たとえばFIXME)

\paragraph{有用物質を作る微生物をデザインするために、どの経路が流れるか予測したい}
広い代謝経路のうち、どこがよく流れるかあらかじめ予測できると、大いに助か
る産業があります。微生物を使って、アミノ酸など人間の生活に役に立つ物質
を生産している企業です。欲しい物質をつくる経路の酵素はより多く発現させ
て流れを太くし、その流れから枝分かれするような経路はせき止めてしまえ、
というように代謝マップとにらめっこしながら、酵素の遺伝子に突然変異を入
れたり遺伝子組換えをしたりしますが、どこに流れているかが予測できなけれ
ば、理詰めでできません。遺伝子1つ1つで試すのは時間も手間もかかりま
す。

\paragraph{代謝経路の配線から、流れの太さを予測する方法がある}
代謝の流れがどこを通っているのかを予測する手法として開発されたのが、こ
の章で説明する代謝流束均衡解析 (Flux Balance Analysis) です。基本となる
枠組みを最初に発表したのは大阪大学の合葉修一教授と松岡正佳助手(現・崇
城大学教授)です。その後、カリフォルニア大学サンディエゴ校のベルンハル
ト・パルソン教授 (Bernhard O. Palsson) らのグループが発展に大きく貢献し
ました。それではまず、代謝流束均衡解析の前提となる3つの概念、「反応速
度」「定常状態」「流束」から説明します。

\section{反応速度、定常状態、流束}


\paragraph{反応速度(reaction rate)}
単位時間あたりに起きる化学反応の回数を体積で割ったもの。

\paragraph{定常状態(steady state)}物質の生成と消費が釣り合っている状態。下の経路図では$v_1 = v_2 = v_3$ となる状態。\(\displaystyle \frac{d[S_1]}{dt}=\frac{d[S_2]}{dt}=0\)とも書ける。
\[\overset{v_1}\rightarrow S_1 \overset{v_2}\rightarrow S_2 \overset{v_3}\rightarrow \]

含まれている物質が一定に保たれた培養液の中で微生物を飼っているとき、細胞内の代謝の流れも一定になり、細胞内物質の生成と消費が釣り合っていると考えます。


\paragraph{平衡状態(equilibrium)}
定常状態とは異なります。上の図では$v_1 = v_2 = v_3 = 0$ に相当する。流れのない、よどんだ池。

\paragraph{流束(flux)}
定常状態における反応速度のことです。慣用的に\(J\)と表記します。下図では\(v_1 = v_2 = v_3 = v_4 = J\) である。
\begin{figure}[h]
\begin{center}
%\includegraphics[scale=0.5]{../Constraint/img/mfa1.tiff}
\includegraphics[keepaspectratio, scale=0.5]{../Constraint/img/mfa1.epsi}
\end{center}
\end{figure}
反応速度は1つの酵素分子の性質ですので、パスウェイ全体からすると局所的な性質です。これに対し、流束は複数の酵素分子にまたがる大域的な性質です。\\


\section{代謝経路の構造は行列で書ける}
ここまでは話を単純にするため、直線の経路で話を進めてきました。実際の代謝経路は分岐や合流が入り組んでいます。行列を使うと、入り組んだ経路でも取り扱いやすくなります。

\paragraph{化学量論係数(stoichiometric coefficient)}
化学反応が1回起こった際に生成・消費される分子数のことです。例えば、${\rm {\bf 2}H}_2 + {\rm ({\bf 1})O}_2 \rightarrow {\rm {\bf 2}H}_2{\rm O}$という反応式の場合、太字で書いた数字を化学量論係数と呼びます。\\

\paragraph{化学量論行列(stoichiometry matrix)}
代謝経路に含まれる全ての化学反応の化学量論係数を並べた行列。代謝経路の
ネットワーク構造に関する全情報を含んでいます。数式中では通常、{\bf N}また
は{\bf S}で表記されます。次の例題で化学量論行列を実際に求めてみます。


\subsection{例題}
次のような経路の化学量論行列を求めてみます。

\begin{figure}[h]
\begin{center}
\includegraphics[scale=0.5]{../MCA/img/mca2.epsi}
\end{center}
\end{figure}

まず、この経路に属するすべての化学反応式を書き出します。\\

\begin{center}
\(
\begin{array}{lrcr}
v_1: & S_1 & \rightarrow & 2S_2 \\
v_2: & S_2 & \rightarrow & S_3 \\
v_3: & 2S_3 & \rightarrow & S_1 
\end{array}
\)
\end{center}

次に、行のラベルが代謝物質、列のラベルが化学反応となるような表をつくります。上の反
応リストを見ると、反応\(v_1\)が1回起きるごとに代謝物質\(S_1\)は1分子だけ減少します。この関係を表すため、\(S_1\)の行と\(v_1\)の列が交差するマス目に「-1」
を書き入れます。\\

\indent 同様にして、反応\(v_1\)が1回起きると基質\(S_2\)が2分子増
えるので、\(S_2\)の行と\(v_1\)の列が交差するマス目に「+2」を書き込みます。ま
た、反応\(v_1\)は基質\(S_3\)の増減には関与していません。このようなときは対応
するマス目に「0」を書きます。
\begin{center}
\(
\begin{array}{r|rrr}
     & v_1 & v_2 & v_3 \\
\hline
S_1: & -1  &   &  \\
S_2: &  2  &   &  \\
S_3: &  0  &   & \\ 
\end{array}
\)
\end{center}

同様にして他のマス目にも、対応する化学量論係数を書き込みます。
完成した表をそのまま行列にしたものが、化学量論行列です。
\begin{center}
\(
\left(
\begin{array}{rrr}
 -1  &  0  & 1 \\
  2  &  -1 & 0 \\
  0  &  1 & -2\\ 
\end{array}
\right)
\)
\end{center}

この経路の基質濃度時間変化は以下のようになります。

\[
\begin{array}{cccc}
\frac{d}{dt}
\left(
\begin{array}{l}
S_1 \\
S_2 \\
S_3 \\ 
\end{array}
\right)
& = &
\left(
\begin{array}{rrr}
 -1  &  0  & 1 \\
  2  &  -1 & 0 \\
  0  &  1 & -2\\ 
\end{array}
\right)
&
\left(
\begin{array}{l}
v_1 \\
v_2 \\
v_3 \\ 
\end{array}
\right)\\
\dot{\bf S} & = & {\bf N} & {\bf v} \\
\end{array}
\]

\section{演習}
次の経路の化学量論行列を求めなさい。各反応の化学量論係数はすべて1とします。
\begin{figure}[h]
\begin{center}
\includegraphics[scale=0.5]{../MCA/img/mca3.epsi}
\end{center}
\end{figure}

\section{零空間(NullspaceまたはKernel)}
行列はベクトルに掛けると、そのベクトルを伸縮・回転させます。伸縮が極端な場合は、ベクトルの長さは0になります。


行列{\bf N}の零空間とは、

\[{\bf NK = 0}\]

を満たす行列{\bf K}のことを指します。

\[{\bf K}\mbox{の列数} = {\bf N}\mbox{の列数}(\mbox{すなわち反応の総数}) – {\rm rank}({\bf N})\]

定常状態は{\bf Nv = 0}ですから、{\bf K}の列は定常状態になる反応速度に対応します。したがって、定常流束{\bf J}は{\bf K}の列要素の線形結合です。

\subsection{例題}
零空間のの求め方

\subsection{演習}
図の経路の零空間を求めなさい。化学量論行列は\({\bf N} = (1\ \ 1\ \ 1)\)です。

\begin{figure}[h]
\begin{center}
\includegraphics[scale=0.5]{../Constraint/img/mfa4.epsi}
\end{center}
\end{figure}

\section{流束分布を求める}
\[{\bf N v_{ss} = 0} \ \ \ (\mbox{または}{\bf S v_{ss} = 0})\]

解は{\bf N}の零空間({\bf K})に存在します。ただし、解が一意に定まらないことが問題点です。そこで、解の選択基準として、菌体の成長を最大にする解を採用することにします。これには実験的な根拠あります (Ibarra et al. 2002 Nature)。数学的に言うと、最適化問題と呼ばれます。

\subsection{目的関数}
線形計画法で解(流束分布)を探します。{\it E. coli}の場合、1gの菌体を形成に必要な流束の比が以下のように定式化されています。これを最大化します。
\[Z_{\rm biomass} = Z_{\rm precursors} + Z_{\rm cofactors}\]
ただし
\begin{eqnarray*}
Z_{\rm precursors} & = & 0.205 v_{G6P} + 0.071 v_{F6P} + 0.898v_{R5P}\\
& + & 0.361 v_{E4P} + 0.129 v_{T3P} + 1.496 v_{3PG}\\
& + & 0.519 v_{PEP} + 2.833 v_{PYR} + 3.748 v_{AcCoA}\\
& + & 1.787 v_{OAA} + 1.079 v_{\alpha KG}\\
& & \\
Z_{\rm cofactors} & = & 42.703 v_{ATP} - 3.547 v_{NADH} + 18.22 v_{NADPH}
\end{eqnarray*}

\subsection{線形計画法の図的解法}
\begin{figure}[h]
\begin{center}
\includegraphics[scale=0.4]{../Constraint/img/mfa5.epsi}
\end{center}
\end{figure}

\subsection{例題}
流束の求め方

\subsection{演習}
\begin{figure}[h]
\begin{center}
\includegraphics[scale=0.4]{../Constraint/img/mfa6.epsi}
\end{center}
\end{figure}

\(v_1\) 、\(v_2\) の最適解を求めなさい。ただし、\(b_1 = b_2 = 10\) mmol / gDW / h 、\(v_1\)、\(v_2\)の最大流束をそれぞれ8 mmol / gDW / h 、6 mmol / gDW / h 、
目的関数は \(Z = 40 v_{\rm ATP} + 3.5 v_{\rm NADH}\)とします。

\subsection{シャドウプライス}
%\section{タイムスケール}
%迅速平衡

定常状態における細胞内の物質の流れ(代謝流束)を予測する。



\section{Further Reading}
\begin{enumerate}
\item 清水先生の本
\item 「代謝工学」
\item Heinrich and Schuster
\item Palsson
\end{enumerate}
