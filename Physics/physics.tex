\chapter{反応速度{\it k}の物理学}
\section{目標}
「活性化エネルギー」を読める
1分子同士の衝突頻度=衝突「確率」、といってよいか?

\section{自由エネルギーとは何か}
\section{熱力学関数}
\subsection{示量・示強}
\subsection{記憶図とLegendre変換}
\section{Boltzmann分布}
\section{Maxwell分布}
\section{前頻度因子の導出}
\subsection{古典衝突理論}
\subsubsection{剛体球衝突モデル}
\section{親和力と反応速度}

\begin{eqnarray*}
Z_t & = & \sigma (v_r) v_r dc_A dc_B\\
\end{eqnarray*}

\(v_r\)分子AとBの相対速度、\(\sigma(v_r)\)分子AとBの衝突断面積。分子Aの半径を\(r_A\)、分子Bの半径を\(r_B\)とすると\(\sigma = 4 \pi (r_A + r_B)^2\)。

ここでMaxwell-Boltzmannの速度分布を用いて、

\subsubsection{反応性衝突モデル}
ただぶつかっただけでは反応しない。2分子反応\(A+B \longrightarrow C+D\)の速度\(v\)は\(v=k[A][B]\)と表される。反応速度は単位体積あたりの反応性衝突に等しい。

\begin{eqnarray*}
v & = & k[A][B]\\
  & = & Z_r\\
  & = & \left\{ \sigma_{AB} \left(\frac{8k_B T}{\pi \mu}\right)^{\frac{1}{2}}{\rm exp}\left(-\frac{\varepsilon_t^{0}}{k_B T}\right) \right\} [A][B]\\
\end{eqnarray*}



\subsection{絶対反応論}

\section{演習: 遷移状態理論と速度定数}
分子動力学(MD)シミュレーションにより、ある反応が起こる前と後での内部エネルギー変化は\(\Delta U = 1.23\)J/molであることが示された。系の温度を310K、エンタルピー変化\(\Delta H \simeq \Delta U\)として次の問いに答えなさい。

\begin{enumerate}
\item エントロピー変化\(\Delta S\)はどのくらいか。\({\displaystyle \Delta S = \frac{\Delta U}{T}}\)である。
\item Eylingの遷移状態理論によれば、速度定数\(k\)は以下の式で表される。MDから求まった\(\Delta H\)、\(\Delta S\)が使えるよう、熱力学の第1法則\(\Delta G = \Delta H - T \Delta S\)を用いて書き直しなさい。
\[k = \frac{k_B T}{h}\kappa e^{-\frac{\Delta G^0}{RT}}\]
\item MDから求まった\(\Delta H\)、\(\Delta S\)を代入して、速度定数\(k\)を求めなさい。
\end{enumerate}


\section{Further reading} 
活性化エネルギー
反応の自由エネルギー変化についてはアイゼンバーグ&クロサーズ
熱力学 キャレン(記憶図)、田崎(ルジャンドル変換)、清水、原島と小出(束縛エネルギー)
統計力学 高校数学で分かるボルツマンの原理、名古屋大の上羽先生のPDF

量子化学計算
吉澤先生の論文
平尾 量子化学計算ビギナーズマニュアル
